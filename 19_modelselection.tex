\section{Model selection}

We often have data on a large number of explanatory variables and wish to build a model using some subset of them. 
The problem becomes choosing between different competing linear models. 
If the model is too small we `underfit' the data. This leads to poor predictions and high bias, but low variance.
If the model is too big we `overfit' the data. This leads to  poor predictions and high variance, but low bias.
When the model is `just right', we balance bias and variance to get good predictions.

One approach towards model selection is to consider all possible subsets of the pool of explanatory variables and find the `best' model according to some predetermined criteria.
Another approach is to use a search algorithm to find the `best' model. The latter is usually more efficient when the number of potential variables is large.
In both cases different criteria may be used to select what constitutes the best model. Popular choices include Adjusted $R^2$, Mallow's $C_p$, AIC and BIC. These criteria assign scores to each model and allow us to choose the model with the best score.


\subsection{Model Selection Criteria}


Before we start, recall that we can partition the data into sums of squares:
$$
\mbox{SST}_{p} = \mbox{SSE}_{p} + \mbox{SSR}_{p}
$$
where
\begin{eqnarray*}
\mbox{SST}_{p} &=& ||\bfy - \bar y {\bf J}_n ||^2 = \bfy' (\bfI - \bfH_{\bf J}) \bfy \\
\mbox{SSE}_{p} &=& ||\bfy - \hat \bfy||^2 = \bfy' (\bfI - \bfH_{\bfX}) \bfy \\\
\mbox{SSR}_{p} &=& ||\hat \bfy - \bfJ_n \bar y||^2 = \bfy' (\bfH_{\bfX} - \bfH_{\bfJ}) \bfy 
\end{eqnarray*}
Here the subscript $p$ refers to the fact that the model includes $p$ parameters.

We can use these values to compute the coefficient of multiple determination, which is given by
$$
R_p^2 = \frac{\mbox{SSR}_{p}}{\mbox{SST}_{p}} = 1- \frac{\mbox{SSE}_{p}}{\mbox{SST}_{p}}.
$$
This represents the proportion of the total variability explained by our model. 
It is guaranteed to take values between $0$ and $1$. High values imply that the explanatory variables are useful in explaining the response and low values imply that the explanatory variables are not useful.
However, since $R_p^2$ increases with the size of the model, it is not a good criterion for variable selection.
It would always choose to include all variables.
Alternative approaches that penalize for unnecessary model complexity are needed.


\subsubsection{Adjusted $R_p^2$}

One such approach is the adjusted coefficient of multiple determination, which uses the mean squares instead of the sums of square, i.e.
$$
R_{a,p}^2 = 1 - \frac{\mbox{MSE}_{p}}{\mbox{MST}_{p}} = 1 - \left( \frac{n-1}{n-p} \right) \frac{\mbox{SSE}_{p}}{\mbox{SST}_{p}}.
$$
Since the term includes the number of model parameters, $p$, it penalizes for model complexity.



\subsubsection{Mallow's $C_p$}

Mallow's $C_p$ is a criteria to assess fits when models with different numbers of parameters are being compared.
It can be expressed as:
$$
C_p = \frac{SSE_p}{s^2} - n + 2p.
$$
Here the MSE of the full model is used to estimate $s^2$.
If the model includes all important variables, then
$$
E(SSE_p) = (n-p) \sigma^2.
$$
If $s^2$ provides a good estimate of $\sigma^2$ then 
$$
E(C_p) \approx \frac{(n-p) \sigma^2}{\sigma^2} -n + 2p = p.
$$
Values close to the corresponding $p$ indicate a good model. 
The smaller the value, the better the model fits.



\subsubsection{Information Criteria}

Many methods are based on combining a term based on the log-likelihood with one based on penalizing model complexity.
This provides a means to balance model fit with model complexity when assessing the best fitting models, providing a way to penalize unnecessarily complicated models.

To illustrate, let $f(\bfy | \bfX, \bfbet)$ be the density of the response $\bfy$.
For a sample of $n$ observations $(\bfx_i, y_i)$, $i=1, \ldots n$,  the log-likelihood is given by
$$
\log (L) = \sum_{i=1}^n log(f(y_i | \bfx_i, \bfbet).
$$
Let $\ell(\hat \beta_p)$ be the log-likelihood evaluated at the MLE under the model with $p$ parameters.
Standard information criteria are of the form:
$$
-2 \ell(\hat \bfbet) - \phi(n,p).
$$
Here the first term represents model fit, and the second a penalized model complexity.
In the linear model setting $\ell(\hat \bfbet) \propto -n \log(SEE_p/n)$.


Akaike's Information Criterion (AIC) tries to balance the conflicting demands of model accuracy and parsimony. 
For the linear model with Guassianity assumption it  can be expressed as:
$$
AIC_p = n \log(SSE/n) + 2p.
$$
here low values indicate a better model.


Several modifications of AIC have been suggested.
For example, the Bayesian Information Criterion (BIC) is defined as:
$$
BIC_p = n \log(SSE/n) + \log(n)p.
$$
Again, low values indicate a better model.

The difference between AIC and BIC lies in the severity of the penalty. 
The penalty is larger for BIC when $n>8$. 
Hence, BIC tends to favor more parsimonious models compared to AIC which has a tendency to overfit (i.e., include too many explanatory variables).


\subsubsection{PRESS}

The prediction sum of squares (PRESS) criterion measures how well the fitted values for a subset model can predict the observed response.
The PRESS statistic is defined as:
$$
PRESS_p = \sum_{i=1}^n (y_i - \hat y_{(i),i})^2
$$ 
Thus, the PRESS statistic is the sum of squared deleted residuals.

The model with the smallest PRESS statistic is considered  `best'. 
Leaving one item out at a time is known as leave-one-out cross-validation.
This allows us to predict the performance of the model on holdout data.


\subsection{Methods for Model Selection}

\subsubsection{Exhaustive Search}

One approach is to consider all possible subsets of the pool of explanatory variables and find the `best' model according to of the criteria outlined above. However, if have $15$ predictors there are $2^{15}$ different models (even before considering interactions, transformations, etc.).


\subsubsection{Step-wise Methods}


When the number of explanatory variables is large it is not feasible to fit all possible models. 
Instead, it is more efficient to use a search algorithm to find the best model.
A number of such algorithms exist, including forward selection, backward elimination and stepwise regression.

Let, us begin by setting up the problem. Assume we are choosing from a set of $P$ possible explanatory variables $v_k,\, k=1,…..P$.
In each algorithm our goal is to find the subset of $v_k$ that best balances model fit and parsimony.
We discuss each algorithm in detail.

\bigskip
\noindent {\bf Forward Selection}

\begin{enumerate}
\item Fit the $P$ simple linear regression models:
$$
y_i = \beta_0 +\beta_1 v_{ki} + \epsilon_i \,\,\, k = 1,\ldots P.
$$

\item Set $x_1=v_k$, where $v_k$ is the variable that has the most significant regression coefficient (i.e. the smallest p-value) If no variable is significant (e.g., none of the p-values are smaller than a preset significance level $\alpha$) the algorithm stops.

\item Lock in the variable $x_1$, and repeat the procedure with models that include two explanatory variables:
$$
y_i = \beta_0 + \beta_1 x_{1i} + \beta_2 v_{ki} + \epsilon_i, \,\,\, k = 1,\ldots P, \,\, v_k \ne x_1.
$$
Set $x_2=v_k$, where $v_k$ is the variable that has the most significant regression coefficient. If no variable is significant stop the algorithm.

\item Continue the procedure until no remaining $v_k$ generate a p-value that is smaller than the preset significance level $\alpha$.
\end{enumerate}

The criteria for choosing whether to include a new variable can vary.
As an alternative to using p-values one can instead use a criteria such as the AIC.
In each step choose the variable whose inclusion lowers the AIC the most. 
If no variables lower the AIC than stop the algorithm.

\bigskip
\noindent {\bf Backward Elimination}

\begin{enumerate}
\item Start by fitting a model that includes all possible variables:
$$
y_i = \beta_0 +\beta_1 v_{1i} + \dots + \beta_Pv_{Pi} +  \epsilon_i .
$$

\item Find the variable $v_k$ which has the least significant regression coefficient (i.e. the largest p-value).
If its p-value is smaller than some preset significance level, stop the algorithm, otherwise drop the variable. 

\item Next, fit the largest model excluding the dropped $v_k$.
Find the variable which has the least significant regression coefficient.
If its p-value is smaller than some preset significance level, stop the algorithm, otherwise drop the variable. 

\item Continue until the algorithm stops.
\end{enumerate}

Alternatively, AIC or BIC can be used as a criteria for determining whether to drop variables.
Start with a full model. In each step choose the variable whose exclusion lowers the AIC the most. 
If the exclusion of any variable does not lower the AIC than stop the algorithm.

\bigskip
\noindent {\bf Stepwise Regression}

\begin{enumerate}
\item Start in the same manner as in forward selection and add the most significant variable from a series of $P$ simple linear regressions. 

\item Once a new variable has been included in the model, check other variables already included in the model for their partial significance. 
Remove the least significant explanatory variable whose p-value is greater than the preset significance level. 

\item Continue until no variables can be added and none removed, according to the specified criteria.
\end{enumerate}

Again, note that AIC can be used instead of p-values.


\subsection{Shrinkage Methods}

The subset selection procedure is a discrete process, as individual variables are either in or out.
This method can have high variance in that a different dataset from the same source can result in a totally different model. Shrinkage methods allow a variable to be partly included in the model. That is, the variable is included but with a shrunken co-efficient. Here popular methods include ridge regression, the Lasso, or the Elastic Net.

