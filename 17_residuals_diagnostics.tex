\section{Residuals and diagnostics}

Now with some distributional results under our belt, we can discuss distributional
properties of the residuals. Note that, as a non-full rank linear transformation of
normals, the residuals are singular normal. When $\by \sim N(\bX \bbeta, \sigma^2 \bI)$, the mean of the residuals is $\bzero$, variance of the residuals is given by
$$
\Var(\be) = \Var\{(\bI - \bH_\bX)\by\} = \sigma^2 (\bI - \bH_\bX).
$$
As a consequence, we see that the diagonal elements of $\bI - \bH_\bX \geq 0$
and thus the diagonal elements of $\bH_\bX$ must be less than one. (A fact that
we'll use later). In addition, we see that although the errors may have equal variance and be uncorrelated the residuals do not.


\subsection{The hat matrix}

Let us study the hat matrix $$\bfH = \bfX (\bfX'\bfX)^{-1} \bfX'$$ in more detail.
Recall it is symmetric and idempotent.
In the continuation we write the $(i,j)^{th}$ element of $\bfH$ as $h_{ij}$.


Since $\mathrm{rank}(\bfH) = \mathrm{trace}(\bfH),$ it follows that
$\ \sum h_{ii} = \tilde{p}$.
If we assume there is an intercept in the model, then $\ \bfH \bfJ_n =
\bfJ_n,\ $ and hence $\ \sum_i h_{ij} = \sum_j h_{ij} = 1$.

\btheo
Assume we have an intercept in the model.  Let ${\cal X}$ be the $n
\times (p-1)$ mean-centered design matrix (without the intercept term).
Let $\ ({\cal X}'{\cal X})_{jk} =
\sum_i(x_{ij}-\bar{x}_j) (x_{ik}-\bar{x}_k)$, and redefine $\bfx_i'$
to be the $i$th row of $\bfX$ without the one for the intercept. Then
the following holds: 
\begin{itemize}
\item[(a)] 
$h_{ii} = \frac{1}{n} + (\bfx_i - \bar{\bfx})' ({\cal X}'{\cal
X})^{-1} (\bfx_i - \bar{\bfx}).\ $ This means in particular that each
$h_{ii}$ is bounded below by $\frac{1}{n}.$
\item[(b)] 
Let $r_i$ be the number of rows of $\bfX$ that are identical to its
$i$th row. Then $\ h_{ii} \leq \frac{1}{r_i}.$
\end{itemize}
\etheo

Note, as $n$ increases $h_{ii}$ tends to decrease.
The term $(\bfx_i - \bar{\bfx})' ({\cal X}'{\cal X})^{-1} (\bfx_i - \bar{\bfx})$ corresponds to a corresponds to a Mahalanobis distance, roughly estimating the distance between $\bfx_i$ and  $\bar{\bfx}$.




\bsexa
Consider the case of simple linear regression.
Here ${\cal X} = (x_1, x_2, \dots x_n)'$.
Then, 
$$\ h_{ii} = \frac{1}{n} + \frac{(x_i-\bar{x})^2}{\sum_k(x_k-\bar{x})^2}\ $$ 
Note $h_{ii} = \frac{1}{n}$ if and only if $x_i = \bar{x}$.
\eexa

\subsection{Studentized Residuals}

It is important to note that the variance of the residuals is not constant, as
$$
Var(e_i) = \sigma^2(1-h_{ii}).
$$ 
Since $h_{ii} \le 1$, the variance will be small if $h_{ii}$ is close to $1$.
In general, this will be true if the observation lies far away from the mean.
This can be problematic, as the model is less likely to hold for these observations.

Another problem with the residuals is that they have the units of $\by$ and thus are
not comparable across experiments. Taking 
$$r_i = \frac{e_i}{s\sqrt{1-h_{ii}}}$$ 
i.e., standardizing the residuals by their estimated standard deviation, solves these problems. However, the resulting quantities are not comparable to
$t$-statistics since the numerator elements (the residuals) are not independent of $s^2$.
The residuals standardized in this way are called ``studentized'' residuals. 
Studentized residuals are a standard part of most statistical software.

\subsection{Coding example}
\begin{verbatim}
> data(mtcars)
> y = mtcars$mpg
> x = cbind(1, mtcars$hp, mtcars$wt)
> n = nrow(x); p = ncol(x)
> hatmat =  x %*% solve(t(x) %*% x) %*% t(x)
> residmat =  diag(rep(1, n)) - hatmat
> e = residmat %*% y
> s = sqrt(sum(e^2) / (n - p))
> rstd = e / s / sqrt(diag(residmat))
> # compare with rstandard, r's function
> # for calculating standarized residuals
> cbind(rstd, rstandard(lm(y ~ x - 1)))
          [,1]        [,2]
1  -1.01458647 -1.01458647
2  -0.62332752 -0.62332752
3  -0.98475880 -0.98475880
4   0.05332850  0.05332850
5   0.14644776  0.14644776
6  -0.94769800 -0.94769800
...
\end{verbatim}

\subsection{PRESS residuals}
\label{sec:press}

Consider the model $\by \sim N(\bW \bgamma, \sigma^2 \bI)$
where $\gamma = [\bbeta' ~ \Delta_i]$, 
$\bW = [\bX ~ \bdelta_i]$ where $\bdelta_i$ is a vector of all zeros
except a 1 for row $i$. This model has a shift in position $i$, for 
example if there is an outlier at that position. This is called the mean-shift outlier model.
The least squares criterion can be written as
\begin{equation}
\label{rstud}
\sum_{k\neq i} \left(y_k - \sum_{j = 1}^p x_{kj} \beta_j \right)^2
+ \left(y_i - \sum_{j=1}^p x_{ij} \beta_j - \Delta_i\right)^2.
\end{equation}
Consider holding $\bbeta$ fixed, then we get that the estimate of
$\Delta_i$ must satisfy
$$
\Delta_i = y_i - \sum_{j=1}^p x_{ij} \beta_j
$$
and thus the right hand term of \eqref{rstud} is 0. Then we obtain $\bbeta$ by minimizing
$$
\sum_{k\neq i} \left(y_k - \sum_{j = 1}^p x_{kj} \beta_j \right)^2.
$$
Therefore $\hat \bbeta$ is exactly the least squares estimate having
deleted the $i^{th}$ data point; notationally, $\hat\bbeta_{(i)}$. Thus, $\hat \delta_i$ 
is a form of residual obtained when deleting
the $i^{th}$ point from the fitting then comparing it to the fitted value, 
$$	
\hat \Delta_i = y_i - \sum_{j=1}^p x_{ij} \hat \beta_{(i),j}.
$$
Notice that the fitted value at the $i^{th}$ data point is then 
$\sum_{j=1}^p x_{ij} \hat \beta_{(i),j} + \hat \Delta_i = \ y_i$ and thus
the residual is zero. The term $\hat \Delta_i$ is called the PRESS residual, the
difference between the observed value and the fitted value with that point deleted.

Since the residual at the $i^{th}$ data point is zero, the estimated variance from
this model is exactly equal to the variance estimate having removed the
$i^{th}$ data point. The $t$ test for $\delta_i$ is then a form of 
standardized residual, that exactly follows a $t$-distribution under the null
hypothesis that $\delta_i = 0$. 

\subsection{Computing PRESS residuals}

It is interesting to note that PRESS residuals don't actually require recalculating the
model with the $i^{th}$ datapoint deleted. Let $\bX' = [\bz_1 ~ \ldots ~ \bz_n]$
so that $\bz_i$ is the $i^{th}$ row of the matrix $\bz$ (hence column $i$ of $\bz'$). 
We use $\bz$ for the rows, since we've already reserved $\bx$ for the columns of $\bX$. 
Notice, then that 
$$
\bfX' \bfX = \sum_{i=1}^n \bz_i \bz_i'.
$$
Thus, $\bfX_{(i)}' \bfX_{(i)}$, the x transpose x matrix with the $i^{th}$ data
point deleted is simply
$$
\bX_{(i)}' \bX_{(i)} = \bX' \bX - \bz_i \bz_i'.
$$
We can appeal to the Sherman, Morrison, Woodbury theorem for the inverse
$$
(\bX_{(i)}' \bX_{(i)})^{-1}
= (\bfX' \bfX)^{-1} + \frac{(\bfX' \bfX)^{-1} \bz_i \bz_i' (\bfX' \bfX)^{-1}}{1 - \bz_i' (\bfX' \bfX)^{-1} \bz_i}
$$
Define $h_{ii}$ as diagonal element $i$ of $\bfX (\bfX' \bfX)^{-1} \bfX'$ which is equal to
$\bz_i' (\bfX' \bfX)^{-1} \bz_i$. (To see this, pre and post multiply this matrix by a 
vector of zeros with a one in the position $i$, an operation which grabs the $i^{th}$ diagonal entry.)
Furthermore, note that $\bX' \by = \sum_{i=1}^n \bz_i y_i$ so that
$$
\bX_{(i)}' \by_{(i)} = \bX' \by - \bz_i y_i.
$$
Then we have that the predicted value for the $i^{th}$ data point where it was not used in the fitting is:
\begin{eqnarray*}
\hat y_{(i),i} & = & \bz_i' (\bX_{(i)}' \bX_{(i)})^{-1} \bX_{(i)}' \by_{(i)}\\
& = & \bz_i' \left((\bfX' \bfX)^{-1} + \frac{(\bfX' \bfX)^{-1} \bz_i \bz_i' (\bfX' \bfX)^{-1}}{1 - h_{ii}} \right)(\bX' \by - \bz_i y_i) \\
& = & \hat y_i + \frac{h_{ii}}{1 - h_{ii}} \hat y_i - h_{ii} y_i - \frac{h_{ii}^2 y_i}{1 - h_{ii}} \\
& = & \frac{\hat y_i}{1 - h_{ii}} + y_i - \frac{y_i}{1 - h_{ii}}
\end{eqnarray*}
So that we wind up with the equality: 
$$
y_i - \hat y_{(i), i} = \frac{y_i - \hat y_i}{1 - h_{ii}} = \frac{e_i}{1 - h_{ii}}
$$
In other words, the PRESS residuals are exactly the ordinary residuals divided by $1 - h_{ii}$. 

\subsection{Externally studentized residuals}
It's often useful to have standardized residuals where a data point in question didn't
influence the residual variance. The normalized
PRESS residuals are, as seen in \ref{sec:press}.  However, the PRESS residuals are 
leave one out residuals, and thus the $i^{th}$ point was deleted for the fitted value. An alternative
strategy is to normalize the ordinary residuals by dividing by a standard deviation estimate
calculated with the $i^{th}$ data point deleted. That is,
$$
\frac{e_i}{s_{(i)}\sqrt{1 - h_{ii}}}.
$$
In this statistic, observation $i$ hasn't had the opportunity to impact the variance estimate.
Note the internally and externally studentized residuals are monotonically
related through
$$t_i = r_i \sqrt{\frac{n-p-1}{n-p-r_i^2}}.$$

Given that the PRESS residuals are $\frac{e_i}{1 - h_{ii}}$, their variance is
$\sigma^2 / \sqrt{1 - h_{ii}}$. Then we have that the press residuals normalized (divided
by their standard deviations) are 
$$
\frac{e_i}{\sigma \sqrt{1 - h_{ii}}}
$$
If we use the natural variance estimate for the PRESS residuals, the estimated variance calculated with
the $i^{th}$ data point deleted, then the normalized PRESS residuals are the same as the
externally standardized residuals. As we know from above, these also arise out of the $t$-test for the
mean shift outlier model from Section \ref{sec:press}. 

\subsection{Coding example}
First let's use the \texttt{swiss} dataset to show how to calculate
the ordinary residuals and show that they are the same as those
output by \texttt{resid}.
\begin{verbatim}
> y = swiss$Fertility
> x = cbind(1, as.matrix(swiss[,-1]))
> n = nrow(x); p = ncol(x)
> hatmat = x %*% solve(t(x) %*% x) %*% t(x)
> ## ordinary residuals
> e = (diag(rep(1, n)) - hatmat) %*% y
> fit = lm(y ~ x)
> ## show that they're equal by taking the max absolute difference
> max(abs(e - resid(fit)))
[1] 4.058975e-12
\end{verbatim}
Next, we calculate the standardized residuals and show how to get
them automatically with \texttt{rstandard}.
\begin{verbatim}
> ## standardized residuals
> s = sqrt(sum(e ^ 2) / (n - p))
> rstd = e / s / sqrt(1 - diag(hatmat))
> ## show that they're equal by taking the max absolute difference
> max(abs(rstd - rstandard(fit)))
[1] 6.638023e-13
\end{verbatim}
Next, let's calculate the PRESS residuals both by
leaving out the $i^{th}$ observation (in this case observation 6)
and by the shortcut formula.
\begin{verbatim}
> i = 6
> yi = y[i]
> yihat = predict(fit)[i]
> hii = diag(hatmat)[i]
> ## fitted model without the ith data point
> y.minus.i = y[-i]
> x.minus.i = x[-i,]
> beta.minus.i = solve(t(x.minus.i) %*% (x.minus.i)) %*% t(x.minus.i) %*% y.minus.i
> yhat.i.minus.i = sum(x[i,] * beta.minus.i)
> pressi = yi - yhat.i.minus.i
> c(pressi, e[i] / (1 - hii))
           Porrentruy 
 -17.96269  -17.96269 
\end{verbatim}
Now show that the \texttt{rstudent} (externally studentized) residuals and normalized
PRESS residuals are the same.
\begin{verbatim}
> ## variance estimate with i deleted
> e.minus.i = y.minus.i - x.minus.i %*% beta.minus.i
> s.minus.i = sqrt(sum(e.minus.i ^ 2) / (n - p  - 1))
> ## show that the studentized residual is the PRESS residual standardized
> ei / s.minus.i / sqrt(1 - hii)
Porrentruy 
 -2.367218 
> rstudent(fit)[i]
        6 
-2.367218 
\end{verbatim}
Finally, show that the mean shift outlier model residuals give the PRESS and
the rstudent residuals.
\begin{verbatim}
> delta = rep(0, n); delta[i] = 1
> w = cbind(x, delta)
> round(summary(lm(y ~ w - 1))$coef, 3)
                  Estimate Std. Error t value Pr(>|t|)
w                   65.456     10.170   6.436    0.000
wAgriculture        -0.210      0.069  -3.067    0.004
wExamination        -0.323      0.242  -1.332    0.190
wEducation          -0.895      0.174  -5.149    0.000
wCatholic            0.113      0.034   3.351    0.002
wInfant.Mortality    1.316      0.376   3.502    0.001
wdelta             -17.963      7.588  -2.367    0.023
\end{verbatim}
So notice that the the estimate for \texttt{wdelta} is the PRESS residual
while the \texttt{t value} is the externally studentized residual.

\subsection{Leverage}

 The term $h_{ii}$ from the hat matrix is called the leverage of the $i^{th}$ observation. The leverage of a observation measures its ability to move the regression model all by itself by simply moving in the $y$-direction. The leverage measures the amount by which the predicted value would change if the observation was shifted one unit in the $y$-direction.
The leverage is always between $0$ and $1$.
A point with close to zero leverage has little effect on the regression model.
If a point has leverage equal to $1$ the line must follow the observation perfectly. 
The $h_{ii}$ depends only on $\bfX$, though a knowledge of $\bfy$ is required for a full interpretation.


\subsubsection{Cook's Distance}

Cook’s distance measures the aggregate influence of the $i^{th}$ value on all $n$ fitted values.
It is defined by
$$ 
D_i = 
\frac{(\hat{\bfbet}_{(i)}-\hat{\bfbet})' \bfX'\bfX
(\hat{\bfbet}_{(i)}-\hat{\bfbet})}{\tilde{p}\hat{\sigma}^2} = 
\frac{(\hat{\bfY}_{(i)}-\hat{\bfY})' 
(\hat{\bfY}_{(i)}-\hat{\bfY})}{\tilde{p}\hat{\sigma}^2},
$$ 
where $\hat{\bfbet}_{(i)}$ are the parameter estimates obtained after
deleting observation $i$, and $\hat{\bfY}_{(i)}$ are the corresponding
fitted values.

An alternative expression for the Cook's distance is  $\ D_i =
\frac{r_i^2}{p} 
\left(\frac{h_{ii}}{1-h_{ii}}\right).$
The value of $D_i$ depends on two functions, the size of the residuals $e_i$ and the leverage value $h_{ii}$. Hence, an observation can be influential by having a large residual and/or a large leverage.
Typically, points with $D_i$ greater than $1$ are classified as influential.
